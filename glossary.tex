% %%-------------Glossary--------------------------------
% \chapter*{Glossary (if any)}
% \addcontentsline{toc}{chapter}{Glossary} 
% %--------------------------Elmi

% \textbf{Object:} An object $o$ is the elementary description of a phenomenon to be studied. The set of objects available for the analysis of the phenomenon is called in this manuscript
% "data-set" and is denoted by $O = {o_1 . . . , o_N}$, where N is the size of the data-set.

% \textbf{Similarity:} The similarity between two objects i and j is denoted $s(i, j)$. Similarity between the objects to be studied is the main (and often unique) information allowing the clustering algorithm to partition the data-set. In most cases, this similarity is calculated according to a "distance" or "dissimilarity" measure adapted to the type of
% data and the problem to be solved. This distance, generally denoted $d(i, j)$, is a measure
% of dissimilarity that satisfies certain mathematical properties. A small similarity
% corresponds of a big value of s and a small value of d. When the data are described in
% a vector space, the distance most used is the Euclidean distance, then denoted .
% $\norm{i-j}$.

% \textbf{Data structure:} When we talk about data "structure" in this thesis, we refer to the
% underlying distribution of the data-set. In particular, a clustering algorithm proposes
% a model of this distribution in the form of a partition of clustered data. In each cluster
% the objects are more similar than objects in different clusters.We call "natural" clusters
% of the data the clusters which represent distinct modes in the distribution of the data:
% the density of "intermediate" objects should be lower than the density of objects in
% each cluster.
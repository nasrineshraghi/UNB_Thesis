%%-------------Abstract-----------------
\doublespacing
\setlength{\parindent}{2em}

\chapter*{Abstract}
% \addcontentsline{toc}{chapter}{Abstract}

% \begin{abstract}
 In the age of Internet of Things, the ability to manage people and devices in indoor spaces has become significantly crucial for developing new applications. %Most of today’s research focuses on analyzing indoor sensor data to reduce costs of service facilities management. 
 Clustering indoor localization data streams has gained popularity in recent years due to an expanding opportunity to discover knowledge and collect insights from data sources. Data clustering offers an intuitive solution to a wide variety of unsupervised classification problems. Clustering solutions are often preferred for problems for which no ground truth is known. Due to the continuous, fast-moving nature of many standard data streams, such as IoT devices and scientific monitoring devices, clustering algorithms are necessary.
 
 
 
 In this thesis, we have developed a data stream clustering algorithm called DSAP for the purpose of analyzing and modeling indoor localization data. The DSAP algorithm is implemented as a two-stage method, online, and offline clustering phases with the landmark time window model.  This algorithm has an improved ability to track the cluster evolution in data streams. The proposed DSAP is non-parametric in the sense of not requiring any prior information about the number of clusters. The validation and efficiency of the DSAP algorithm are evaluated using intrinsic indices and time and space complexity metrics to analyze two indoor localization datasets. 
% \end{abstract}





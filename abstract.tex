%%-------------Abstract-----------------
\doublespacing
\setlength{\parindent}{2em}

%\chapter*{Abstract}
\addcontentsline{toc}{chapter}{Abstract}

\begin{abstract}
 In the age of Internet of Things, the ability to manage people and devices in indoor spaces has become significantly crucial for developing new applications. %Most of today’s research focuses on analyzing indoor sensor data to reduce costs of service facilities management. 
 Clustering indoor localization data streams has gained popularity in recent years due to an expanding opportunity to discover knowledge and collect insights from data sources. Data clustering offers an intuitive solution to a wide variety of unsupervised classification problems. Clustering solutions to problems often arise in areas in which no ground truth is known. Due to the continuous, fast-moving nature of many standard data streams, such as IoT devices and scientific monitoring devices, clustering algorithms are necessary.
 
 
 
 In this thesis, we developed a novel data stream clustering method called DSAP with the applicative purpose of modeling indoor localization data. The DSAP algorithm is implemented with a two-stage, online, and offline clustering method with the landmark time window model.  This algorithm with an improved ability to track the cluster evolution in data streams is desirable, at least for reasons. The proposed DSAP is non-parametric in the sense of not requiring any prior information about the number of clusters. The validation and efficiency of the DSAP algorithm are evaluated using intrinsic indexes and time and space complexity metrics. In the end, the results have been obtained from DSAP are compared with the well-known streaming implementation, Streaming K-mean.    
\end{abstract}





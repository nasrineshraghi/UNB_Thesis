%%-------------Abstract-----------------
\doublespacing
\setlength{\parindent}{2em}

%\chapter*{Abstract}
% \addcontentsline{toc}{chapter}{Abstract}

\begin{abstract}
 In the age of Internet of Things, the ability to find spatio-temporal patterns of people and devices moving in indoor spaces has become crucial for developing new applications. In particular, clustering indoor localization data streams has gained popularity in recent years due to their potential of generating relevant information  for  planning  building  automation, evaluating energy efficiency scenarios, and simulating emergency protocols. In this thesis, a data stream Affinity Propagation (DSAP) clustering algorithm is proposed for analyzing indoor localization data generated from e-counters and WiFi localization systems. The data sets are a sequence of potentially infinite and non-stationary data streams, arriving continuously where random access to the data is not feasible and storing all the arriving data is impractical.  The DSAP algorithm is implemented based on a two-phase approach (i.e. online and offline clustering phases) using the landmark time window model. The proposed DSAP is non-parametric in the sense of not requiring any prior knowledge about the number of clusters and their respective labels. The validation and performance of the DSAP algorithm are evaluated using real-world data streams from two experiments aimed at finding stair usage patterns and occupancy behaviour in indoor spaces. 

%  intrinsic indices and time and space complexity metrics to analyze two indoor localization datasets.
 
 
 
 
%  Streaming cluster analysis is often preferred for problems for which no ground truth is known. Due to the continuous, fast-moving nature of many standard data streams, such as IoT devices and scientific monitoring devices, clustering algorithms are necessary.
 
%  an expanding opportunity to discover new insights  insights from data sources. Data clustering offers an intuitive solution to a wide variety of unsupervised classification problems. 
 
 
 
%  In this thesis, we have developed a data stream clustering algorithm called DSAP for the purpose of analyzing and modeling indoor localization data. The DSAP algorithm is implemented as a two-stage method, online, and offline clustering phases with the landmark time window model.  This algorithm has an improved ability to track the cluster evolution in data streams. The proposed DSAP is non-parametric in the sense of not requiring any prior information about the number of clusters. The validation and efficiency of the DSAP algorithm are evaluated using intrinsic indices and time and space complexity metrics to analyze two indoor localization datasets. 
\end{abstract}





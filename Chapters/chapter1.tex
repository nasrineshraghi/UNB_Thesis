%%-----------Chapters start-------------------------------------
%%-----------Chapter 1------------------------------------------
% \documentclass[../UNBThesis2.tex]{subfiles}
\setlength{\parindent}{2em}
% \begin{document}
\chapter{Introduction}
%\setcounter{secnumdepth}{3} \pagenumbering{arabic}
%\setcounter{page}{1} \pagestyle{myheadings}
%\markboth{}{}\markright{} \rhead{\thepage} \setcounter{page}{1}
%\pagestyle{myheadings} \pagenumbering{arabic} \rhead{\thepage}
%\setcounter{page}{1}

%%%%%%%%%%%%%%%%%%%%%%%%%%%%%%%%%%%%%%%%
% Good summary for the introduction
%Data stream affinity propagation algorithm has been chosen for indoor localization data in this research work. For this algorithm, every data point is a centroid, and it means they have equal importance and there is no prior reason. Also, the AP algorithm does not require the number of clusters as an input which makes our work more robust, especially if it uses for the cloud side. AP's main issue is the computation efficiency that we make it possible by applying the time window model and making it a stream approach. This algorithm is applied in many scientific research and different datasets such as intrusion detection, energy, gene expression, psychology, business, physical science, social science, etc. However, to the best of our knowledge AP model has not yet been applied for any people counting data ,and streaming AP has been never used for any indoor localization dataset. To the best of our knowledge,  streaming AP models have not yet been applied for computing micro-clusters using people counting data streams in indoor spaces
%%%%%%%%%%%%%%%%%%%%%%%

%Indoor positioning and occupancy models are an emerging area of interest. An exponential growth in the usage of indoor non-intrusive sensors and data interpretation algorithms has been reported in the last few years. The DSAP algorithm is proposed as a robust and efficient method to analyze indoor positioning and building occupancy datasets. 
%Many real-world applications require the knowledge of the user location in order to provide relevant services. Hence, many research efforts are being made to develop automatic user localization tools. The studies conducted in this work are another step in the direction. The user's position and attributes are estimating by using measurements from electronic devices or sensors. 


A data stream is a real-time, continuous, ordered (implicitly by arrival time or explicitly by timestamp) series of data points coming at a very high speed. Data streaming involves two main issues. The first one is processing the fast incoming data. Because of its infinite nature and rate, it is impossible to store the data and analyze it in an offline mode. From its inception, data streaming faces large-scale issues, and new algorithms to interpret them, e.g., clustering, classification are required.
The second concern is to deal with the variations in the underlying data distribution due to the evolution of the phenomenon.

Clustering is one of the most important unsupervised learning methods which partitions a given set of data points into subsets called clusters, such that data points in the same cluster are similar and data points in different clusters are dissimilar. Clustering data stream continuously produces and maintains the clustering structure from the data stream.

Time window models including sliding, damped, landmark, and pyramidal can be used within the stream clustering algorithms  \cite{nguyen2015survey}. These models aim to handle the evolution of the distribution of the data stream over time. Applying a time window makes it possible to analyze and store a stream within a specific time frame and discard the previous historical data \cite{mansalis2018evaluation}.

Previous research has proven that partitioning-based clustering models are suited for problems where the optimal number of micro-clusters is a-priori unknown. The data processing is optimized by representative a cluster-head for each micro-cluster. Cluster-heads of micro-clusters have been previously explored for improving indoor fingerprinting using Wi-Fi RSS data \cite{hu2015improving, subedi2019improving}. They have also been successfully applied for improving routing schemes for multi-level heterogeneous Wireless Sensor Networks, reducing energy consumption \cite{wang2019affinity}. 

Affinity propagation algorithm has been chosen for indoor localization data in this research work. Affinity propagation (AP) clustering algorithm is a partitioning-based message passing algorithm that treats every data point as a potential centroid, and each point is given equal importance \cite{dueck2009affinity}. Also, the AP algorithm does not require the number of clusters as an input which makes any continuously operating automated operation more robust, especially if it used on a cloud. It has been successfully applied in many scientific research and for different datasets such as intrusion detection, energy estimation, gene expression, psychology, business, physical science, social science, etc. 
Affinity propagation unfortunately also suffers from high memory complexity, and hence it's ability to handle large datasets diminishes rapidly with size and cannot handle massive data sets. To solve this problem, a stream clustering algorithm is developed using a time window model in this work. There has not been a lot of research work using stream clustering methods for analyzing indoor localization data, especially affinity propagation. No research work on indoor localization data streams using AP algorithm has been reported previously to the best of our knowledge.

This thesis proposes a novel Data Stream Affinity Propagation (DSAP) algorithm using the landmark time window model for clustering indoor localization data streams obtained from two experiments deployed in indoor spaces. DSAP Algorithm is capable of supporting the online-offline phases. In the online phase, micro-clusters are constantly computed using a landmark time window model to handle the most recent data in the stream and to continuously follow the changing data distribution. The offline phase is then performed, and the micro-clusters themselves are clustered to provide the overall clustering results. The entire online and offline phases that deliver the final clusters are performed without any user intervention.

Indoor positioning and occupancy models are an emerging area of interest. An exponential growth in the usage of indoor non-intrusive sensors and data interpretation algorithms has been reported in the last few years. 
% Indoor positioning technologies have been a critical enabler for advancing the usefulness of the Internet of Things in biomedical and health care applications.
Many positioning technologies have been explored for generating positioning information of people in indoor spaces, including Wi-Fi, BLE beacons, RFID tags, visible light wave, and ultra-wide band \cite{namiot2015indoor, jeon2018ble}. In particular, infra-red, optical (e.g., RGB videos or flat images), break beam, thermal, and ultrasonic sensors have been used for counting people in indoor spaces \cite{mautz2012indoor}. 

The DSAP algorithm is used to find patterns in two distinct indoor localization datasets. 
The experimental results on the localization datasets validate the algorithm's effectiveness in handling dynamically evolving data streams. The evaluation of DSAP clustering results consists of two phases, intrinsic validation, and clustering performance evaluation. Clustering performance evaluation uses execution time and memory usage of the DSAP algorithm, which are important efficiency factors for streaming algorithms. 

% Moreover, data stream AP models usually suffer from a quadratic computational complexity in the number N of items. %The time complexity of AP is $O(n^2*T)$, where $n$ is the number of samples and $T$ is the number of iterations until convergence happens\cite{refianti2017time}. 
% This computational performance might become an issue for computing macro-clusters using large volumes of people counting data. Strategies are important to be developed to improve the performance of AP models. 


% A data stream is a real-time, continuous, ordered (implicitly by arrival time or explicitly by timestamp) sequence of items arriving at a very high speed [Golab and ¨Ozsu, 2003].
% Data streaming involves two main issues [Aggarwal, 2007]. The first one
% is processing the fast incoming data; because of its amount and pace, there
% is no way to store the data and analyze it offline. From its inception data
% streaming faces large scale issues and new algorithms to achieve e.g., clustering,
% classification, frequent pattern mining, are required.
% The second issue is to deal with the changes in the underlying data distribution,
% due to the evolution of the phenomenon under study (the traffic, the
% users, the modes of usage, and so forth, can evolve). The proposed approach
% aims at solving both issues, by maintaining a model of the data coupled with
% a change detection test: as long as no change in the underlying data distribution
% is detected, the model is seamlessly updated; when the change detection
% test is triggered, the model is rebuilt from the current one and the stored
% outliers.




% Clustering refers to partitioning a set of observations or tuples into groups according to some desired criterion. So, intra-cluster objects are similar and inter-cluster objects are dissimilar. Mining pattern in data streams is increasing in many applications such as smart grids, sensor networks, financial,  medical data, network  data,  and so on. 
% As illustrated in figure \ref{dataclu} shows different clustering approaches nowadays are applying with different techniques.

% \begin{figure}
% \centering%\cite{An2013}
% \includegraphics[width = 12cm,height = 7cm]{image/111.png}
% \caption{Comparing different clustering Algorithms }
% \label{dataclu}
% \end{figure}



%%---------------GHALAT ELMI-----------------------------------------------
% Streams consist of data tuples that need to be processed as they arrive, and mining these streams is challenging since the data distribution underlying
% a stream can evolve significantly over time. Besides dealing with evolving distributions, stream clustering algorithms have to meet several technical requirements, including limited time, limited memory, and processing the stream in a single pass. A multitude of stream clustering algorithms has been proposed in the literature that satisfy these requirements. 

% Of major importance is the quality of the resulting clustering, which can be measured by evaluation measures, also termed criteria, indices, validation measures, or validation indices.

%%-----------------------------------------------------------




\section{Research Objectives}

The overall research goal is to extend the current streaming AP models to generate micro and macro clusters from indoor localization data. The objectives can be described as follows. 

\begin{itemize}

    \item Develop the data stream AP model for uncovering evolutionary patterns with two-stages, online micro and offline macro phases.
     \item Applying landmark time window model to find the latest data points in the stream with an expiration number to keep the latest data in the model.
    \item Evaluate the clustering results with intrinsic indices and performance evaluation to find the advantages and limitations of the proposed algorithm.  
    %\item Analyze the behavioral pattern in indoor localization experiments.
    \item Demonstrate the potential of DSAP for analyzing localization data streams to understand staircase usage patterns as well as occupant patterns in indoor spaces.
\end{itemize}  
    % \item Apply the online-offline phases of such a model to analysing indoor positioning stream data.  
    % \item Applying two data related to indoor localization positioning 
    %has certainly not been applied 
    % \item Apply streaming AP clustering algorithm to compare the proposed model and evaluate our model.

    %\item The proposed method is based on unsupervised learning, and data clustering. Affinity Propagation (AP) is a clustering message passing algorithm proposed by Frey and Dueck. This algorithm is selected for its properties of stability and of the way it represented each data cluster. The consequence of using AP is quadratic computational complexity, especially on large scale datasets.




 

%\section{Research Questions}


%The research questions of this research work are: 
% \begin{itemize}
%     \item Is streaming AP more suitable than streaming K-means for analysing e-counter data? 
%     \item Maintain a continuously consistent good clustering of the sequence observed so far, using a small amount of memory and time.

% \end{itemize}


\section{Scientific Contributions}


The main scientific contribution of this research work summarized as follow:

\begin{itemize}
    %\item  extend a partitional clustering technique that can adapt to appearing or disappearing concepts in underlying data, thus achieving similar or better quality and performance of clustering than less-flexible techniques without requiring a specific level of clustering to use. NOT CLEAR WHAT YOU MEANT HERE
    
    \item A new data stream clustering algorithm for covering evolutionary patterns using landmark time window model is proposed. 
    \item  The new streaming AP clustering algorithms is used for clustering indoor localization data streams for the first time.
    
    \item We demonstrate the potential of DSAP in improving our understanding of stair usage and occupant behaviour in indoor spaces datasets.
\end{itemize}


%\todo[inline]{}research premises 
%As far as we had research, there is not any research work present with affinity propagation used landmark time window model.


\section{Organization of this Thesis}
This thesis is organized in five chapters as described as follows:

Chapter 2 provides a background of cluster analysis.

Chapter 3 summarizes the related work in streaming AP algorithms.

Chapter 4 describes the proposed DSAP algorithm for data stream clustering.

Chapter 5 includes a summary of implementation and the data sets applied for this research work.

Chapter 6 provides an in-depth analysis of the data and a discussion of the clustering results.

Finally, Chapter 7 concludes the work and outlines future research work.
% \end{document}






%There are challenges to designing a data stream platform, such as unbounded data or no control over the data arrival.

%%%%%%%%%%%%%%%%%%%%%%%%%%%%%%%%%%%%%%%
% Although AP still does not guarantee global optimum, several experiments in [13] have shown its consistent superiority over the previous algorithms. However, AP clustering has a limitation that it is hard to determine the value of parameter ‘preference’, which can lead to a suboptimal clustering solution.
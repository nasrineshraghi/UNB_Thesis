% \documentclass[../UNBThesis2.tex]{subfiles}
\setlength{\parindent}{2em}

% \begin{document}
\chapter{Conclusions and Future Work}
%%%%%%%%%%%%%%%%%%%%%%%%%%%%%%%%%%%%%%%%%%%%%%%%%%%%%%%%%%%%%
% summary
% a more detailed summary of the work

% \begin{itemize}

%     \item Develop the data stream AP model for uncovering evolutionary patterns with two-stages, online micro and offline macro phases.
%      \item Applying landmark time window model to find the latest data points in the stream with an expiration number to keep the latest data in the model.
%     \item Evaluate the clustering results with intrinsic indices and performance evaluation to find the advantages and limitations of the proposed algorithm.  
%     %\item Analyze the behavioral pattern in indoor localization experiments.
%     \item Demonstrate the potential of DSAP for analyzing localization data streams to understand staircase usage patterns as well as occupant patterns in indoor spaces.
% \end{itemize} 

% also compare how DSAP clustering these two different data sets

% emphasize the scientific contributions


% \begin{itemize}
  
%     \item A new data stream clustering algorithm for covering evolutionary patterns using landmark time window model is proposed. 
%     \item  The new streaming AP clustering algorithm is used for clustering indoor localization data streams for the first time.
    
%     \item We demonstrate the potential of DSAP in improving our understanding of stair usage and occupant behaviour in indoor spaces datasets.
% \end{itemize}

% advantages and limitations
% future research work
%%%%%%%%%%%%%%%%%%%%%%%%%%%%%%%%%%%%%%%%%%%%%%%%%%%%%%%%%%%%%%%%%%%%
\section{Summary}

With the increasing number of IoT devices, a staggering amount of data streams are being generated in a continuous flow. These data streams consist of a sequence of data points that need to be analyzed using processing techniques that do not have access to all data. In this research work, the DSAP model was developed for uncovering evolutionary patterns based on two phases: online micro and offline macro phases. The landmark time window model was proposed to ensure that only the latest data points in the stream are used in the clustering process.

The concept of having expiring data points within a time-window repository was introduced in this research work. The expiration time is a hyperparameter that can be changed based on the different needs of the users. Moreover, the latest landmark time window keeps the micro-cluster centroids of the model updated and the size of the micro-clusters under control. The combination of the time windows and the expiration time enables the DSAP model to handle concept drift and changing data distribution over time.

In the online phase, the arriving data points were analyzed in near real-time and relevant summary statistics were provided. The results of the online phase consisted of a number of micro-clusters that represented a large number of preliminary clusters in the stream. In the offline phase, the micro-clusters were re-clustered to generate a final set of macro-clusters. The number of macro-clusters was always much smaller than the number of micro-clusters.

The clusters found using the proposed DSAP clustering algorithm were evaluated by applying intrinsic validation indices for unlabeled data such as silhouette index, Caliński-Harabasz index, and Davies-Bouldin index. These metrics were calculated for both the micro and macro-clusters by estimating the centroids for each phase and data points related to each of the centroids. In addition, the performance of the algorithm was evaluated by using the time and space complexity metrics. These metrics have proven that the implementation of the DSAP algorithm is a robust stream clustering approach capable of handling data streams generated by indoor localization systems commonly used in IoT.

The existing AP-based stream clustering models have not been applied for finding clusters from indoor localization data streams prior to this work. Two distinct localization data streams were analyzed using the DSAP algorithm. Preliminary results demonstrate the strength of the algorithm by finding meaningful patterns and providing new insights into the indoor localization data based on the number of people and their locations in an indoor space. 


The first data set analyzed was the e-counter data from an experiment performed at Flinders University to capture the impact of a motivational intervention on the physical activity of people. The feature space consisted of two attributes: the number of people passing by a sensor and the position of each sensor at different stairs in the building. Hourly and monthly stair usage patterns were successfully captured using the landmark time window of one-hour intervals. The clustering results showed the behaviour of people during the three months, i.e., before, during, and after the intervention. Some positive changes in behavioural patterns were observed during some days of the experiment and at specific levels of the building. Therefore, it can be concluded that the intervention campaign had a weak positive impact on people.
%This experiment was not successful in motivating people using stairs, but we were able to observe some changes in behavioural patterns. 

The second implementation of DSAP was carried out using the WiFi indoor localization data streams in order to find out the occupancy behaviour in three buildings at the Universitat Jaume I campus. The features chosen for the DSAP algorithm were SPACEIDs that represent the occupancy of a particular classroom, lab, or office, and the PHONEIDs that represent the participants. The algorithm was robust in capturing the occupancy behaviour of the users using a landmark time window of ten minutes. 
 
The overall clustering results were compared with the standard AP clustering algorithm. DSAP performed better than the standard AP model since a major improvement in the processing time was achieved. DSAP was able to reduce the processing time by a couple of orders of magnitude as compared with the AP algorithm. The memory and processing time also demonstrated that DSAP can easily handle continuous data streams.

The metrics of assessing the quality of DSAP clusters such as Silhouette, Caliński\-Harabasz, and Davies\-Bouldin indexes show that the accuracy of micro-clusters are improved, while the same results did not observe for macro-clusters and it is due to the spread clusters.

The main advantage of the DSAP model relies on its simple and straightforward approach but still retaining the strengths of other streaming AP-based algorithms (i.e., StrAP, IStrAP, ISTRAP, and APdenstream) while removing some of the complexities introduced by them. The combination of a user-defined expiration time, a dynamic threshold $\epsilon_{W_j}$, the landmark time window model, and a time-window repository have been crucial to developing a manageable, fast and accurate clustering algorithm. DSAP runs entirely autonomously without the need to specify the number of clusters once the hyperparameters are selected. 


Due to stream data availability, the DSAP model was not implemented using live streaming data. Therefore, the main limitation of the DSAP model is that it does not handle latency and bandwidth problems, which can occur very often when analyzing live streaming data.  Stream data clustering of multiple variables is also expected to bring scaling issues. Highly dynamic data with large concept drift is expected to reduce the performance of the DSAP algorithm as new clusters will be formed in almost every window slowing down the entire process. These limitations are planned to be addressed in future versions of the algorithm.

% The proposed DSAP algorithm is a simplified model of the StrAP algorithm with the improvement in threshold and expiration value. This algorithms does not require number of clusters to be set before running the algorithm and it is advantage especially for back-end and cloud infrastructure. DSAP has some limitation that needs to be improved such as how well it can detect outliers. 

%Also, it tested in a small environment and if the space increase, the small clusters will not be detected anymore by applying an adaptive threshold.   

%The code was tested with the landmark time windows but any window model can be used depending on the problem under study. 
\section{Future Work}

The DSAP model opens up exciting new avenues for future work not only on real-time data analytics across various sectors but also in model development and implementation. The code is freely available, and its modular framework enables the easy addition of new features. For example, another distance function such as Manhattan distance can be easily used to compute the threshold $\epsilon_{W_j}$ without changing the DSAP source code. The shape of the clusters is influenced by choice of the distance function. Multiple distance functions can be used to find the optimal function depending on the data distribution and shape of the clusters. However, changing the landmark time window model to another time window model needs further research to understand the impact in finding macro-clusters. It is crucial to carefully select the time window model taking into account the type of data streams of an application.

The performance of the DSAP model needs to be tested using real-world data streams. Future research work will focus on improving the DSAP model for handling missing or out of order data packets. We will also explore the effects of scalability and noise on the performance and robustness of the DSAP algorithm.

Finally, a benchmark test would be interesting to evaluate the DSAP model against the existing streaming AP models using standardized data sets. The main challenge is that the source code of most of these previous models is not available. 

% Another interesting study would be to benchmark the performance of DSAP algorithm with other stream clustering algorithms using standardized data sets. Additionally, the effect of the shape and kind of the data distribution on the algorithm needs to be quantified.

%The work done in this thesis shows the potential of using DSAP for live monitoring of indoor occupancy and localization. Intriguing new insights can be obtained from analyzing long term continuously operating IOT systems leading to better planning and services. We will continue to develop the model further to incorporate more features and test it on multiple datasets to make it a robust tool that produces accurate models for indoor occupancy.  

% The very important part which remained is experimenting the algorithm on another real data stream. We evaluated the model on two real data but with the simulation of the stream to find the flaws and weaknesses better.

% One specific feature of the presented work is to deal with application domains where a cluster of items cannot easily be represented by an average/artifact item, although the distance or similarity between any two items can be computed. 

% Another interesting research direction would be handling high dimension data. Euclidean distance-based algorithm cannot handle well with high-dimensional data streams. Potential future work would be to expand DSAP to overcome these difficulties.

% advantage
% limitation  DSAP with data without outlier \ small environment concentrated are, but with large and spreat - compute time window upto 10 second conform 
% scale ability --simulated  
% not test real world data stream
% come cross latency -bandwidth during stream -- not deal with  feature research 
% how is it new?based on strap -simplified . with no issue of outlier - epsilon to reduce the complexity of (contribution) 


%%%%%%%%%%%%%%%%%%%%%%%%%%%%%%%%%%%%%%%%%%%%%%%%%%%%%%%%%%%%%%%%%%%%%%%%%%%%%%%%%%%%%%%%

% Last but not least, building an architecture on the cloud would be interesting.

%compare DSAP with STRAP
% #I will work later on : 
% other types of windows
% distribution test
% use real world data stream or benchmarking
% try different distances 
% swclustering
% macro cluster user define
% why not the current algo? we have k-means streaming but this is mean of data --> not useful for out data , we cant have mean of levels or people
% macro --> end of stream but we need user request
% indoor movement predict
% algo --> if the number of micro cluster increase , weshould forget the old data
% algo --> if the window end AP (but if there is no point?)
% macro phase is not time sensetive. all micro clusters have same value(DCStream)
% how is indoor positioning calculated
% reoccurence detection(collect inactive clusters)
% why ap? this algo works based on the distances. which is the best fit for our dataset. we havel levels and people. we cant use kmeansto give us a average distanfe which may be between 2 levels!
% compare DSAP with other data srtream algorithm and diffrent dataset is future work
% apply dsap with diffrent distances

% DSAP still has some drawbacks such as handlind the old cluster and not through them out.
% data distribution change by PH test or any other test
% % \end{document}
% good examples for stream clustering and clustering

%%%%%%%%%%%%%%%%%%%%%%%%%%%%%%%%%%%%%%%%%%%%%%%%%%%%%%%%%%%%%%%%%%%%%%%%%%%%%%%%

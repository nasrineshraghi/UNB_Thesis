% \documentclass[../UNBThesis2.tex]{subfiles}
\setlength{\parindent}{2em}

% \begin{document}
\chapter{Conclusions and Future Work}
%%%%%%%%%%%%%%%%%%%%%%%%%%%%%%%%%%%%%%%%%%%%%%%%%%%%%%%%%%%%%
% summary
% a more detailed summary of the work

% \begin{itemize}

%     \item Develop the data stream AP model for uncovering evolutionary patterns with two-stages, online micro and offline macro phases.
%      \item Applying landmark time window model to find the latest data points in the stream with an expiration number to keep the latest data in the model.
%     \item Evaluate the clustering results with intrinsic indices and performance evaluation to find the advantages and limitations of the proposed algorithm.  
%     %\item Analyze the behavioral pattern in indoor localization experiments.
%     \item Demonstrate the potential of DSAP for analyzing localization data streams to understand staircase usage patterns as well as occupant patterns in indoor spaces.
% \end{itemize} 

% also compare how DSAP clustering these two different data sets

% emphasize the scientific contributions


% \begin{itemize}
  
%     \item A new data stream clustering algorithm for covering evolutionary patterns using landmark time window model is proposed. 
%     \item  The new streaming AP clustering algorithm is used for clustering indoor localization data streams for the first time.
    
%     \item We demonstrate the potential of DSAP in improving our understanding of stair usage and occupant behaviour in indoor spaces datasets.
% \end{itemize}

% advantages and limitations
% future research work
%%%%%%%%%%%%%%%%%%%%%%%%%%%%%%%%%%%%%%%%%%%%%%%%%%%%%%%%%%%%%%%%%%%%
\section{Summary}

With the increase in the number of internet-of-things devices, a staggering amount of dynamic streaming data is generated on a continual basis. This data needs to be processed sequentially using stream processing techniques that do not have access to all of the data. In this research work a data stream affinity propagation clustering algorithm called DSAP was proposed to analyze streaming data. The DSAP model was developed for uncovering evolutionary patters in two stages, online micro and offline macro phases using the landmark time window model that ensured that only the latest data points in the stream are chosen.

The concept of expiring clusters was introduced for the latest landmark window which kept the centroids of the model updated and kept the size of the micro-clusters  under control. This expiration value is a hyper-parameter that can be changed based on the needs of the user. The combination of the time windows and the expiration parameter enables the model to handle concept drift and changing data distribution over time.

In the online phase arriving data points are evaluated in near real-time and relevant summary statistics are provided. The result of the online phase are a number of micro-clusters that represent a large number of preliminary clusters in the stream. In the offline phase, the micro-clusters are re-clustered to derive a final set of macro-clusters. The number of macro-clusters is much smaller than the number of micro-clusters.

The outputs of the proposed DSAP clustering algorithm are evaluated by applying intrinsic indices for unlabeled data such as silhouette index, Caliński-Harabasz index, and Davies-Bouldin index. These metrics were calculated for both the micro and macro-clusters by estimating the centroids for each phase and data points related to each of the centroids. In addition, the performance of the algorithm is evaluated by using the time and space complexity metric. 


The DSAP model is a generic stream analysis tool capable of handling any form of streaming data. It is especially useful for extracting summary information from a large number of indoor localization and internet of things sensors. AP based stream clustering models had not been employed for localization prior to this work. Two distinct localization data streams were analyzed using the DSAP algorithm. Preliminary results demonstrate the strength of the algorithm by finding meaningful patterns and provide insights into the indoor localization data based on the number of people and their locations. 


The first dataset analyzed was the e-counter data from an experiment performed at Flinders University to capture the impact of a motivational intervention on the physical activity of people. The key parameters used to determine the clusters were the number of people and the position of each sensor. Hourly features in the data were successfully captured using a one hour windows. The clustering results from the whole experimental period summarized the pattern of people's movement during the three months, before, during and after the intervention. Even though some changes in behavioural patterns was observed but, it can not be conclusively said if the intervention experiment had a positive impact or not.

%This experiment was not successful in motivating people using stairs, but we were able to observe some changes in behavioural patterns. 

The second implementation of DSAP was done on the WiFi indoor localization data to find out the occupancy behaviour due to the moving participants from one building to another in the three buildings at the Universitat Jaume I campus. The features chosen for the DSAP algorithm were SPACEIDs that represents the location of each room or offices and the PHONEIDs that represents the participants. The algorithm was quite agile in capturing the dynamic movement of the users using a landmark time window of ten minutes. 
 
 
Both data set's results were evaluated and compared with the standard AP clustering algorithm in terms of cluster quality and performance. The DSAP clustering results show slightly better metrics for the micro-cluster phase but, a major improvement in the processing time was achieved. DSAP was able to reduce the processing time by a couple of orders of magnitude as compared with the AP algorithm. The memory and processing time show that DSAP can easily handle extensive data sets. 

The main contribution of this work is the development of a simple data stream clustering algorithm which retains the strengths of the other AP based algorithms (StrAP, IstrAP, ISTRAP, and APdenstream) while removing some of the complexities introduced by them. The combination of a dynamic threshold, AP on cached data every window and expiration time creates a light weight, agile and accurate clustering algorithm. The algorithm runs completely autonomously without the need to specify the number of clusters once the hyper-parameters are set. 

The model was tested with two indoor localization datasets and the performance was found to be satisfactory. Due to time limitations the model couldn't be tested on a live streaming data. The efficacy of the entire system in a real world environment would depend on how the algorithm handles latency and bandwidth issues. Data clustering on multiple variables is expected to be bring in scaling issues. Highly dynamic data with large concept drift is expected to reduce the efficacy of the system as new clusters will be formed in almost every window slowing down the entire process. All of these issues are planned to be addressed in the future versions of the algorithm.

% The proposed DSAP algorithm is a simplified model of the StrAP algorithm with the improvement in threshold and expiration value. This algorithms does not require number of clusters to be set before running the algorithm and it is advantage especially for back-end and cloud infrastructure. DSAP has some limitation that needs to be improved such as how well it can detect outliers. 

%Also, it tested in a small environment and if the space increase, the small clusters will not be detected anymore by applying an adaptive threshold.   


\section{Future Work}

The DSAP model opens up exciting new avenues for future work not only on real-time data analytics across various sectors but also in model development and efficiency. The code is freely available online and it's modular framework enables easy addition of new features. For example, any idea for calculation of the dynamic threshold can be incorporated just by replacing the threshold function without changing the rest of the code. The code was tested with the landmark time windows but any window model can be used depending on the problem under study. The shape of the clusters is influenced by the choice of the distance function. Multiple distance functions can be used to find the optimal function depending on the data distribution and shape of the clusters.

The model's performance needs to be extensively tested with real streaming/cloud based data. Improvements might have to be made to make it capable of handling missing or out of sequence data packets. Future research will explore the effects of scalability and noise on the efficiency and robustness of the algorithm. Another interesting study would be to benchmark the performance of DSAP algorithm with other stream clustering algorithms using standardized data sets. Additionally, the effect of the shape and kind of the data distribution on the algorithm needs to be quantified.

The work done in this thesis shows the potential of using DSAP for live monitoring of indoor occupancy and localization. Intriguing new insights can be obtained from analyzing long term continuously operating IOT systems leading to better planning and services. We will continue to develop the model further to incorporate more features and test it on multiple datasets to make it a robust tool that produces accurate models for indoor occupancy.  

% The very important part which remained is experimenting the algorithm on another real data stream. We evaluated the model on two real data but with the simulation of the stream to find the flaws and weaknesses better.

% One specific feature of the presented work is to deal with application domains where a cluster of items cannot easily be represented by an average/artifact item, although the distance or similarity between any two items can be computed. 

% Another interesting research direction would be handling high dimension data. Euclidean distance-based algorithm cannot handle well with high-dimensional data streams. Potential future work would be to expand DSAP to overcome these difficulties.

% advantage
% limitation  DSAP with data without outlier \ small environment concentrated are, but with large and spreat - compute time window upto 10 second conform 
% scale ability --simulated  
% not test real world data stream
% come cross latency -bandwidth during stream -- not deal with  feature research 
% how is it new?based on strap -simplified . with no issue of outlier - epsilon to reduce the complexity of (contribution) 


%%%%%%%%%%%%%%%%%%%%%%%%%%%%%%%%%%%%%%%%%%%%%%%%%%%%%%%%%%%%%%%%%%%%%%%%%%%%%%%%%%%%%%%%

% Last but not least, building an architecture on the cloud would be interesting.

%compare DSAP with STRAP
% #I will work later on : 
% other types of windows
% distribution test
% use real world data stream or benchmarking
% try different distances 
% swclustering
% macro cluster user define
% why not the current algo? we have k-means streaming but this is mean of data --> not useful for out data , we cant have mean of levels or people
% macro --> end of stream but we need user request
% indoor movement predict
% algo --> if the number of micro cluster increase , weshould forget the old data
% algo --> if the window end AP (but if there is no point?)
% macro phase is not time sensetive. all micro clusters have same value(DCStream)
% how is indoor positioning calculated
% reoccurence detection(collect inactive clusters)
% why ap? this algo works based on the distances. which is the best fit for our dataset. we havel levels and people. we cant use kmeansto give us a average distanfe which may be between 2 levels!
% compare DSAP with other data srtream algorithm and diffrent dataset is future work
% apply dsap with diffrent distances

% DSAP still has some drawbacks such as handlind the old cluster and not through them out.
% data distribution change by PH test or any other test
% % \end{document}
% good examples for stream clustering and clustering

%%%%%%%%%%%%%%%%%%%%%%%%%%%%%%%%%%%%%%%%%%%%%%%%%%%%%%%%%%%%%%%%%%%%%%%%%%%%%%%%

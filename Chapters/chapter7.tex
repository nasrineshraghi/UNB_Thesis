% \documentclass[../UNBThesis2.tex]{subfiles}
\setlength{\parindent}{2em}

% \begin{document}
\chapter{Conclusions and Future Work}

This research work was proposed a data stream affinity propagation clustering algorithm called DSAP with the implementation of the landmark time window model with the two stage, online-micro and offline-macro clustering. The DSAP model is capable of handling any form of streaming data. It is specially useful for extracting summery information from a large number of indoor localization and internet of things sensors which they have not been applied on AP stream clustering models.
The proposed DSAP algorithm is validated the clustering results based on the intrinsic clustering validation and performance evaluation metrics.  

Our preliminary results demonstrate the impact they have on finding meaningful patterns and provide insight into indoor localization data based on number of people and their locations. The first data, e-counter, is an experiment performed in Flinders university to capture the impact of motivational intervention on the physical activity of people. The clustering results shows the patter of people movement during three month, before, during and after intervention. This experiment was not successful to motivate people using stairs.

The second implementation of DSAP was on the WiFi indoor localization data to show the concept-drift and people movement from one building to another and how DSAP can detect this pattern.  

Future research will explore other time window models such as sliding, damped and pyramidal models for the DSAP algorithm. Moreover, adding weight to macro cluster to show more accurate representation of all micro cluster with the importance would be the next step.  

The very important part which is remained is experiment the algorithm on another real data stream. We evaluated the model on two real data but with the simulation of stream to find the flaws and weaknesses better.

One specific feature of the presented work is to deal with application domains where a cluster of items cannot easily be represented by an average/artifact item, although the distance or similarity between any two items can be computed. 


% #I will work later on : 
% other types of windows
% distribution test
% use real world data stream or benchmarking
% try different distances 
% swclustering
% macro cluster user define
% why not the current algo? we have k-means streaming but this is mean of data --> not useful for out data , we cant have mean of levels or people
% macro --> end of stream but we need user request
% indoor movement predict
% algo --> if the number of micro cluster increase , weshould forget the old data
% algo --> if the window end AP (but if there is no point?)
% macro phase is not time sensetive. all micro clusters have same value(DCStream)
% how is indoor positioning calculated
% reoccurence detection(collect inactive clusters)
% why ap? this algo works based on the distances. which is the best fit for our dataset. we havel levels and people. we cant use kmeansto give us a average distanfe which may be between 2 levels!
% compare DSAP with other data srtream algorithm and diffrent dataset is future work
% apply dsap with diffrent distances

% DSAP still has some drawbacks such as handlind the old cluster and not through them out.
% data distribution change by PH test or any other test
% % \end{document}
% good examples for stream clustering and clustering
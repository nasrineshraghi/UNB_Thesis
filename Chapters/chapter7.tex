% \documentclass[../UNBThesis2.tex]{subfiles}
\setlength{\parindent}{2em}

% \begin{document}
\chapter{Conclusions and Future Work}
%%%%%%%%%%%%%%%%%%%%%%%%%%%%%%%%%%%%%%%%%%%%%%%%%%%%%%%%%%%%%%
% summary
% a more detailed summary of the work

% \begin{itemize}

%     \item Develop the data stream AP model for uncovering evolutionary patterns with two-stages, online micro and offline macro phases.
%      \item Applying landmark time window model to find the latest data points in the stream with an expiration number to keep the latest data in the model.
%     \item Evaluate the clustering results with intrinsic indices and performance evaluation to find the advantages and limitations of the proposed algorithm.  
%     %\item Analyze the behavioral pattern in indoor localization experiments.
%     \item Demonstrate the potential of DSAP for analyzing localization data streams to understand staircase usage patterns as well as occupant patterns in indoor spaces.
% \end{itemize} 

% also compare how DSAP clustering these two different data sets

% emphasize the scientific contributions


% \begin{itemize}
  
%     \item A new data stream clustering algorithm for covering evolutionary patterns using landmark time window model is proposed. 
%     \item  The new streaming AP clustering algorithm is used for clustering indoor localization data streams for the first time.
    
%     \item We demonstrate the potential of DSAP in improving our understanding of stair usage and occupant behaviour in indoor spaces datasets.
% \end{itemize}

% advantages and limitations
% future research work
%%%%%%%%%%%%%%%%%%%%%%%%%%%%%%%%%%%%%%%%%%%%%%%%%%%%%%%%%%%%%%%%%%%%
\section{Summary}
This research work was developed a data stream affinity propagation clustering algorithm called DSAP with the implementation of the landmark time window model with the two-stage, online-micro and offline-macro clustering. The landmark time window model is capable of handling concept drift and changing the data distribution over time. Also, by introducing an expiration value for the latest landmark windows, the DSAP is able to keep updated centroids and control the number of them not to grow beyond the control. This value is user define and can change based on the needs of the user.

The online phase evaluates arriving data points in real-time and captures
relevant summary statistics. A result is a number of micro-clusters that represent a large number of
preliminary clusters in the stream. In the offline phase, micro-clusters are re-clustered to derive a final set of macro-clusters. The number of macro-clusters is much smaller than the number of micro-clusters.

The proposed DSAP clustering outputs is validated by applying intrinsic indices such as silhouette index, Caliński-Harabasz index, and Davies-Bouldin index when ground true label of data is not available. These metrics were calculated for both micro and macro-clusters with knowing the centroids for each phase and data points related to each of centroids.
In addition, the performance of the algorithm is evaluated by using the time and space complexity metric. 


The DSAP model is capable of handling any form of streaming data. It is especially useful for extracting summary information from a large number of indoor localization and internet of things sensors that have not been applied on AP stream clustering models.

Our preliminary results demonstrate the impact they have on finding meaningful patterns and provide insight into indoor localization data based on number of people and their locations. The first data, e-counter, is an experiment performed in Flinders university to capture the impact of motivational intervention on the physical activity of people. The key parameters used to determine the clusters were number of people and the position of each sensor were able to provide meaningful context on what the clusters represent based on hourly interval. The clustering results shows the patter of people movement during three month, before, during and after intervention. This experiment was not successful to motivate people using stairs, but we were able to observed some changes in behavioural patterns. 

The second implementation of DSAP was on the WiFi indoor localization data to find out the occupancy behaviour by moving participant from one building to another for the three building at the Universitat Jaume I campus. The ten minutes time interval for the data it gave us 

Both data sets results were evaluated and compared with AP standard clustering algorithm in terms of cluster quality and performance. The DSAP clustering results has slightly improved for micro-cluster phase. The memory and processing time show that DSAP can easily handle very large data sets. 

% The micro and macro-clusters of these two data sets generated by the DSAP algorithm are compared with the standard affinity propagation algorithm and both results show that the performance of proposed algorithm is higher than the AP algorithm.



\section{Future Work}

Future research will explore other time window models such as sliding, damped and pyramidal models for the DSAP algorithm.  

The very important part which is remained is experiment the algorithm on another real data stream. We evaluated the model on two real data but with the simulation of stream to find the flaws and weaknesses better.

One specific feature of the presented work is to deal with application domains where a cluster of items cannot easily be represented by an average/artifact item, although the distance or similarity between any two items can be computed. 

Another interesting research direction would be handling high dimension data. Euclidean distance-based algorithm, cannot handle well with high-dimensional data streams. Potential future work would be to expand DSAP to overcome these difficulties.

% Last but not least, building an architecture on the cloud would be interesting.
% advantage
% limitation  DSAP with data without outlier \ small environment concentrated are, but with large and spreat - compute time window upto 10 second conform 
% scale ability --simulated  
% not test real world data stream
% come cross latency -bandwidth during stream -- not deal with  feature research 





% #I will work later on : 
% other types of windows
% distribution test
% use real world data stream or benchmarking
% try different distances 
% swclustering
% macro cluster user define
% why not the current algo? we have k-means streaming but this is mean of data --> not useful for out data , we cant have mean of levels or people
% macro --> end of stream but we need user request
% indoor movement predict
% algo --> if the number of micro cluster increase , weshould forget the old data
% algo --> if the window end AP (but if there is no point?)
% macro phase is not time sensetive. all micro clusters have same value(DCStream)
% how is indoor positioning calculated
% reoccurence detection(collect inactive clusters)
% why ap? this algo works based on the distances. which is the best fit for our dataset. we havel levels and people. we cant use kmeansto give us a average distanfe which may be between 2 levels!
% compare DSAP with other data srtream algorithm and diffrent dataset is future work
% apply dsap with diffrent distances

% DSAP still has some drawbacks such as handlind the old cluster and not through them out.
% data distribution change by PH test or any other test
% % \end{document}
% good examples for stream clustering and clustering


how is it new?based on strap -simplified . with no issue of outlier - epsilon to reduce the complexity of (contribution) 
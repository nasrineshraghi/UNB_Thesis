\documentclass[../UNBThesis2.tex]{subfiles}
\setlength{\parindent}{2em}

\begin{document}
\chapter{Conclusion and Future Work}

This research work was proposed a novel affinity propagation data stream clustering algorithm called DSAP with the implementation of the landmark time window model for keeping the updated data points from the stream. The DSAP model is capable of handling any form of streaming data. It is specially useful for extracting summery information from a large number of indoor localization and internet of things sensors which they have not been applied on AP stream clustering models.
The proposed DSAP algorithm was compared with the established streaming K-means algorithm and validate the clustering results based on the intrinsic clustering validation and performance evaluation metrics. These metrics which are 

Future research will explore other time window models such as sliding, damped and pyramidal models for the DSAP algorithm. Also, the event-based DSAP model is presented but we have not test it on these two datasets and next step will test our model with this format.

The next step for improvement DSAP is to having arcit


One specific feature of the presented work is to deal with application domains where a cluster of items cannot easily be represented by an average/artifact item, although the distance or similarity between any two items can be computed. I

other types of windows
detecting distribution
use real world data stream or benchmarking
try different distances 
swclustering
macro cluster user define
why not the current algo? we have k-means streaming but this is mean of data --> not useful for out data , we cant have mean of levels or people
macro --> end of stream but we need user request
indoor movement predict
algo --> if the number of micro cluster increase , weshould forget the old data
algo --> if the window end AP (but if there is no point?)
macro phase is not time sensetive. all micro clusters have same value(DCStream)
how is indoor positioning calculated
reoccurence detection(collect inactive clusters)
why ap? this algo works based on the distances. which is the best fit for our dataset. we havel levels and people. we cant use kmeansto give us a average distanfe which may be between 2 levels!
compare DSAP with other data srtream algorithm and diffrent dataset is future work
apply dsap with diffrent distances

DSAP still has some drawbacks such as handlind the old cluster and not through them out.
data distribution change by PH test or any other test
\end{document}
